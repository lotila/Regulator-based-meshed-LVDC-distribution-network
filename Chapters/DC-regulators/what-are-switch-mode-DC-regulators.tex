Switch-mode regulators, on the other hand, use switching technology to control the output voltage. They work by rapidly switching the input voltage on and off, and using a feedback loop to control the switching frequency and duty cycle. Switch-mode regulators are more efficient than linear regulators and can handle larger input-to-output voltage differences, but they are more complex and can generate electromagnetic interference (EMI).

Overall, switch-mode regulators are a useful and versatile technology that offers many benefits for a wide range of electronic applications. They are used in a variety of applications, including power supplies for electronic devices, battery chargers, and voltage converters for automotive and industrial use. Due to their high efficiency and compact size, switch-mode regulators are becoming increasingly popular in portable devices and in applications where space is at a premium.

There are two different switch-mode regulators that could work in meshed network, which are step-down converter and chopper converter.



\begin{comment}

potentiaal sources
"Power Electronics: Converters, Applications, and Design" by Ned Mohan, Tore M. Undeland, William P. Robbins.
"Pulse Width Modulation for Power Converters: Principles and Practice" by Paul C. C. Hwang
"Switching Power Converters: Design and Analysis" by Khalid Sheikh.

\end{comment}