A step-down converter, also known as a buck converter, is a type of switch-mode power converter that is designed to convert a higher input voltage to a lower output voltage. This is achieved by rapidly switching a device, such as a transistor, on and off to regulate the output voltage.

The basic circuit of a step-down converter includes an inductor, a switch, a diode and a capacitor. The inductor stores energy during the on-time of the switch and releases energy to the output during the off-time of the switch. The diode is used to prevent the inductor from discharging energy back to the input. The capacitor is used to filter the output voltage.

Step-down converters have several advantages over traditional linear regulators, including high efficiency, small size and wide input voltage range, which makes them widely used in various electronic devices such as personal computers, mobile phones, tablets, and other portable devices.

However, step-down converters also have some disadvantages, such as the potential for EMI and the need for additional components, such as inductors and capacitors, to filter the output voltage.





\begin{comment}

additional sources
https://en.wikipedia.org/wiki/Buck_converter
"Switching Power Supply Design" by Abraham I. Pressman
"Design of Switching Power Supplies, UPS and Voltage Stabilizers" by Slawomir Tumanski
"Power Electronics: Converters, Applications, and Design" by Ned Mohan, Tore M. Undeland, William P. Robbins.

\end{comment}