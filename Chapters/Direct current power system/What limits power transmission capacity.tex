Power transmission in DC transmission lines are limited by two factors, power line thermal limitation and voltage stability.

Power line thermal limitation refers to the maximum amount of power that can be transmitted over a power line without causing damage to the line or creating unsafe conditions. The thermal limit of a power line is determined by the maximum temperature that the line and its insulation can withstand without suffering damage.

The thermal limit of a power line is affected by several factors, including the size and material of the conductor, the type and thickness of the insulation, the ambient temperature, the altitude. The thermal limit of a power line is usually determined by the manufacturer and is specified in the technical data sheet for the line.

Exceeding the thermal limit of a power line can cause damage to the line and create unsafe conditions. Therefore, it is important to ensure that the thermal limit of a power line is not exceeded during normal and abnormal operating conditions. This can be done by monitoring the temperature of the line and the ambient conditions, and by controlling the amount of power transmitted over the line.

Voltage stability in DC power transmission systems refers to the ability of the system to maintain a stable voltage level at all buses under all times.  Variations in the load demand or in the power generation should not have significant effect on voltage level. Power systems have voltage level margin, for example in Finland it is +-5\% of the nominal voltage.

Voltage stability is crucial for the operation of the DC power transmission systems because it affects the ability of the system to transfer power and maintain power quality.







\begin{comment}

potential sources
[1] J. Grainger and W. Stevenson, Power System Analysis, McGraw-Hill, 1994.
[2] T.A. Short, Power System Stability and Control, IEEE Press, 1993.
[3] J.K. Saini and P. Kundur, Power System Stability and Control, CRC Press, 2007.
[4] R. Lasseter and G. Andhankar, "Microgrids and Active Distribution Networks," IEEE Power and Energy Magazine, vol. 11, no. 3, pp. 40-50, 2013.
\end{comment}